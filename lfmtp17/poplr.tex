\documentclass[preprint]{sigplanconf}
%\documentclass{sigplanconf}
% \usepackage[total={155mm,200mm},
%             top=40mm,
%             left=30mm]{geometry}
%\usepackage[T1]{fontenc}
%\usepackage{microtype}
%\usepackage{fix-cm}
%\usepackage[hidelinks, bookmarks, draft=false]{hyperref}
\usepackage{amsmath}
\usepackage{amsthm}
\usepackage{amssymb}
\usepackage{amsfonts}
%\usepackage[final]{listings}
%\usepackage{lstextract}
% \usepackage{graphicx}
\usepackage{graphics}
\usepackage{srcltx}
% \usepackage{charter}
% \usepackage{euler}
%\usepackage{enumerate}
%\usepackage{latexsym}
%\usepackage{comment}
\usepackage{xargs}
%\usepackage[dvipsnames,svgnames]{xcolor}
\usepackage{proof}
\usepackage{url}
\usepackage{xspace}
\usepackage{natbib}
\usepackage{cdsty}
% \usepackage[obeyDraft,colorinlistoftodos]{todonotes}
% \setlength{\marginparwidth}{2.7cm} % as per section 1.6.7 of the todonotes manual
%\usepackage[answerdelayed,lastexercise]{exercise}
%\usepackage{appendix}
%\usepackage{makeidx}

%\makeindex

% ---------------------------------------------------------------------------
% ------------------------------ Todo commmands -----------------------------
% ---------------------------------------------------------------------------

% \newcommandx{\unsure}[2][1=]{\todo[linecolor=red,backgroundcolor=red!25,bordercolor=red,#1]{#2}}
% \newcommandx{\change}[2][1=]{\todo[linecolor=blue,backgroundcolor=blue!25,bordercolor=blue,#1]{#2}}
% \newcommandx{\info}[2][1=]{\todo[linecolor=OliveGreen,backgroundcolor=OliveGreen!25,bordercolor=OliveGreen,#1]{#2}}
% \newcommandx{\improvement}[2][1=]{\todo[linecolor=Plum,backgroundcolor=Plum!25,bordercolor=Plum,#1]{#2}}


% ---------------------------------------------------------------------------
% ---------------------------- Names of languages ---------------------------
% ---------------------------------------------------------------------------

\newcommand{\beluga}{\textsc{Beluga}\xspace}
\newcommand{\twelf}{\textsc{Twelf}\xspace}


% ---------------------------------------------------------------------------
% ------------------------ Theorems and environments ------------------------
% ---------------------------------------------------------------------------

\newtheorem{@problem}{Exercise}[section]
\newenvironment{problem}{\begin{@problem}\rm}{\end{@problem}}
\newtheorem{@sol}{Solution}[section]
\newenvironment{sol}{\begin{@sol}\rm}{\end{@sol}}
\newtheorem{@axiom}{Axiom}
\newenvironment{axiom}{\begin{@axiom}\rm}{\end{@axiom}}

\newtheorem{definition}{Definition}[section]
\newtheorem{theorem}{Theorem}[section]
\newtheorem{conjecture}[theorem]{Conjecture}
\newtheorem{corollary}[theorem]{Corollary}
\newtheorem{proposition}[theorem]{Proposition}
\newtheorem{lemma}[theorem]{Lemma}


% % ---------------------------------------------------------------------------
% % ------------------------------ Contextual ML ------------------------------
% % ---------------------------------------------------------------------------

% \lstdefinelanguage{ContextualML}
% {
%   morekeywords={and, block, case, of, mlam, fn, impossible, let, in, schema,
%     some, rec, type, ctype, prop, stratified, inductive, coinductive, LF, if, then,
%     else, total},
%   keepspaces=true,
%   sensitive,
%   morecomment=[l]{\%},
%   morecomment=[n]{\%\{}{\}\%},
%   morestring=[b]"
% }[keywords,comments,strings]

% \lstloadlanguages{ContextualML}
% \lstset{language=ContextualML}

% % The order of the "literate" definitions is significant:
% %   later definitions shadow earlier ones.  The \\Pi definition must come
% %   *after* the \\ definition, or the first part of \\Pi --- that is, \\ --- will
% %   be matched, and instead of $\Pi$ you'll get $\lambda Pi$.
% %
% \lstset{literate={->}{{$\rightarrow~$}}2 %
%                  {=>}{{$\Rightarrow~$}}2 %
%                  {|-}{{$\vdash\,$}}2 %
%                  {..}{{$.\hspace{-0.025cm}.\hspace{-0.025cm}.$}}1 % is there any nicer way?
%                  {\\}{{$\lambda$}}1 %
%                  {\\Pi}{{$\Pi$}}1 %
%                  {\\gamma}{{$\gamma$}}1 %
%                  {\\psi}{{$\psi$}}1 %
%                  {\\sigma}{{$\sigma$}}1 %
%                  {FN}{{$\Lambda$}}1 %
%                  {<<}{\color{ForestGreen}}1 %
%                  {<<r}{\color{FireBrick}}1 %
%                  {<*}{\color{ForestGreen}}1 %
%                  {<dim}{\color{DimGrey}}1 %
%                  {>>}{\color{black}}1 %
%                  {?}{\bf{?}}1,
%         columns=[l]fullflexible,
%         basicstyle=\ttfamily\lst@ifdisplaystyle\footnotesize\fi,
%         keywordstyle=\bf,
%         identifierstyle=\relax,
%         stringstyle=\relax,
%         commentstyle=\slshape\color{DimGrey},
%         breaklines=true,
%         % breakatwhitespace=true,   % doesn't do anything (?!)
%         mathescape=true,   % interprets $...$ in listing as math mode
%         xleftmargin=0.5cm,
%       }


% ---------------------------------------------------------------------------
% ----------------------- Standard math/CS notations ------------------------
% ---------------------------------------------------------------------------

\newcommand{\subtype}{\leq}

\newcommand{\union}{\mathrel{\cup}}
\newcommand{\sect}{\mathrel{\cap}}
\newcommand{\unit}{\texttt{()}}
\newcommand{\bang}{\texttt{!}}
\renewcommand{\gets}{\mathop{\texttt{:=}}}

\newcommand{\down}{\mathrel{\,\Downarrow\,}}
\newcommand{\step}{\mathrel{\,\Rightarrow\,}}

\newcommand{\syn}{\mathrel{\,\Rightarrow\,}}
\newcommand{\chk}{\mathrel{\,\Leftarrow\,}}
\newcommand{\arr}{\mathrel{\texttt{->}}}
\newcommand{\entails}{\vdash}
\newcommand{\such}{~|~}

\newcommand{\unif}{\doteq}
\newcommand{\totp}{\Rightarrow}
\newcommand{\emp}{\emptyset}

\newcommand{\sectty}{\mathrel{\text{\&}}}

\newcommand{\B}{{\mathcal{B}}}
\newcommand{\D}{{\mathcal{D}}}
\newcommand{\C}{{\mathcal{C}}}
\newcommand{\E}{{\mathcal{E}}}
\newcommand{\F}{{\mathcal{F}}}
\newcommand{\V}{{\mathcal{V}}}
\newcommand{\W}{{\mathcal{W}}}
\renewcommand{\S}{{\mathcal{S}}}

\newcommand{\edot}{\bullet}

\newcommand{\shiftn}[2]{\uparrow^{#1}\!#2}
\newcommand{\shift}[1]{\shiftn{}{#1}}
\newcommand{\app}{\;}

\renewcommand{\bnfas}{\;\mathrel{::=}\;}
\renewcommand{\bnfalt}{\, \mid \,}
\newcommand{\lamdb}{\lam\;} % Lambda for de Bruijn


% ---------------------------------------------------------------------------
% ------------------------ Judgments, properties, types ---------------------
% ---------------------------------------------------------------------------

\newcommand{\Int}{\textsf{int}}
\newcommand{\Float}{\textsf{float}}
\newcommand{\Bool}{\textsf{Bool}}
\newcommand{\Real}{\textsf{real}}
\newcommand{\String}{\textsf{string}}
\newcommand{\Char}{\textsf{char}}
\newcommand{\Nat}{\textsf{Nat}}
\newcommand{\Unit}{\textsf{unit}}
\newcommand{\Ref}{~\textsf{ref}}
\newcommand{\Array}{~\textsf{array}}

\newcommand{\Value}{~\mathsf{value}}
\newcommand{\NumValue}{~\textsf{num-value}}
\newcommand{\Halts}{~\mathsf{halts}}
\newcommand{\Steps}{\mathrel{\,\longrightarrow\,}}
\newcommand{\MSteps}{\longrightarrow^*}
\newcommand{\BSteps}{\Downarrow}
\newcommand{\Translates}{\leadsto}
\newcommand{\ShiftBy}[3]{\mathsf{shift}\;#1\;#2\;#3}

\newcommand{\FV}{\mathsf{FV}}
\newcommand{\FMV}{\mathsf{FMV}}


% ---------------------------------------------------------------------------
% ------------------------ Terms for object languages -----------------------
% ---------------------------------------------------------------------------

\newcommand{\tmtrue}{\textsf{true}}
\newcommand{\tmfalse}{\textsf{false}}
\newcommand{\tmif}[3]{\textsf{if\;} #1 \textsf{\;then\;} #2 \textsf{\;else\;} #3}
\newcommand{\tmfun}[3]{\textsf{fun } #1 (#2) = #3}
\newcommand{\tmfn}[2]{\textsf{fn } #1\;\texttt{=>}\;#2}
\newcommand{\tmrectyp}[3]{\textsf{rec } {#1}\,:\,{#2}\;\texttt{=>}\;#3}
\newcommand{\tmrec}[2]{\textsf{rec } {#1}\texttt{=>}\;#2}
\newcommand{\tmlet}[3]{\textsf{let } #1 = #2 \textsf{\;in\;} #3\; \textsf{end}}

\newcommand{\tmapp}[2]{\mathsf{app}\;#1\;#2}
\newcommand{\tmlam}[3]{\mathsf{lam}\;#1{:}#2.#3}
\newcommand{\tmhastype}[2]{#1 : #2}

\newcommand{\tmarr}[2]{\mathsf{arr}\;#1\;#2}

\newcommand{\tmfst}[1]{\textsf{fst}\;{#1}\xspace}
\newcommand{\tmsnd}[1]{\textsf{snd}\;{#1}\xspace}

\newcommand{\tmzero}{\textsf{z}}
\newcommand{\tmsucc}[1]{\textsf{succ}~#1}
\newcommand{\tmpred}[1]{\textsf{pred}~#1}
\newcommand{\tmiszero}[1]{\textsf{iszero}~#1}

\newcommand{\tmeq}[2]{\mathsf{eq}\;#1\;#2}


% ---------------------------------------------------------------------------- %
% -------------------------- Inference rules' names -------------------------- %
% ---------------------------------------------------------------------------- %

\newcommand{\ruleName}[1]{\textsc{\footnotesize #1}\xspace}

\newcommand {\TIf}           {\ruleName{T-If}}
\newcommand {\TPred}         {\ruleName{T-Pred}}
\newcommand {\TSucc}         {\ruleName{T-Succ}}
\newcommand {\TZero}         {\ruleName{T-Zero}}
\newcommand {\TIsZero}       {\ruleName{T-Iszero}}
\newcommand {\TPlus}         {\ruleName{T-Plus}}
\newcommand {\TMult}         {\ruleName{T-Mult}}
\newcommand {\TEq}           {\ruleName{T-Eq}}
\newcommand {\TApp}          {\ruleName{T-App}}
\newcommand {\TLam}          {\ruleName{T-Lam}}
\newcommand {\TAbs}          {\ruleName{T-Abs}}
\newcommand {\TSub}          {\ruleName{T-Sub}}
\newcommand {\TFn}           {\ruleName{T-Abs}}
\newcommand {\TFun}          {\ruleName{T-Fun}}
\newcommand {\TPair}         {\ruleName{T-Pair}}
\newcommand {\TFst}          {\ruleName{T-Fst}}
\newcommand {\TSnd}          {\ruleName{T-Snd}}
\newcommand {\TVar}          {\ruleName{T-Var}}
\newcommand {\TNum}          {\ruleName{T-Num}}
\newcommand {\TTrue}         {\ruleName{T-True}}
\newcommand {\TFalse}        {\ruleName{T-False}}
\newcommand {\TBase}         {\ruleName{T-Base}}

\newcommand {\TBinaryPrimop} {\ruleName{T-Binary-Primop}}
\newcommand {\TUnaryPrimop}  {\ruleName{T-Unary-Primop}}
\newcommand {\TTuple}        {\ruleName{T-Tuple}}
\newcommand {\TTupleSyn}     {\ruleName{T-Tuple-Syn}}
\newcommand {\TRec}          {\ruleName{T-Rec}}
\newcommand {\TAnno}         {\ruleName{T-Anno}}

\newcommand {\TrLam}         {\ruleName{Tr-Lam}}
\newcommand {\TrApp}         {\ruleName{Tr-App}}
\newcommand {\TrTop}         {\ruleName{Tr-Top}}
\newcommand {\TrNext}        {\ruleName{Tr-Next}}

\newcommand {\TLet}          {\ruleName{T-Let}}
\newcommand {\TLetSyn}       {\ruleName{T-Let-Syn}}
\newcommand {\TDecs}         {\ruleName{T-Decs}}
\newcommand {\TByName}       {\ruleName{T-By-Name}}
\newcommand {\TByVal}        {\ruleName{T-By-Val}}
\newcommand {\TByValTuple}   {\ruleName{T-By-Val-Tuple}}

\newcommand {\EIfTrue}       {\ruleName{E-IfTrue}}
\newcommand {\EIfT}          {\EIfTrue}                          % to be deleted
\newcommand {\EIfFalse}      {\ruleName{E-IfFalse}}
\newcommand {\EIfF}          {\EIfFalse}                         % to be deleted
\newcommand {\EIf}           {\ruleName{E-If}}
\newcommand {\ESucc}         {\ruleName{E-Succ}}
\newcommand {\ESuc}          {\ESucc}                            % to be deleted
\newcommand {\EPred}         {\ruleName{E-Pred}}
\newcommand {\EPredZero}     {\ruleName{E-PredZero}}
\newcommand {\EPredZ}        {\EPredZero}                        % to be deleted
\newcommand {\EPredSucc}     {\ruleName{E-PredSucc}}
\newcommand {\EIszero}       {\ruleName{E-Iszero}}
\newcommand {\EIsZero}       {\EIszero}                          % to be deleted
\newcommand {\EIszeroZero}   {\ruleName{E-IszeroZero}}
\newcommand {\EIsZeroZ}      {\EIszeroZero}                      % to be deleted
\newcommand {\EIszeroSucc}   {\ruleName{E-IszeroSucc}}
\newcommand {\EIsZeroSucc}   {\EIszeroSucc}                      % to be deleted

\newcommand {\EAppFnStep}    {\ruleName{E-App1}}
\newcommand {\EAppArgStep}   {\ruleName{E-App2}}
\newcommand {\EAppBeta}      {\ruleName{E-App-Abs}}
\newcommand {\EAbsStep}      {\ruleName{E-Abs}}

\newcommand {\MRef}          {\ruleName{M-Ref}}
\newcommand {\MTr}           {\ruleName{M-Tr}}
\newcommand {\MStep}         {\ruleName{M-Step}}
\newcommand {\MOne}          {\ruleName{M-One-Step}}

\newcommand {\NVZero}        {\ruleName{Nv-Zero}}
\newcommand {\NVSucc}        {\ruleName{Nv-Succ}}

\newcommand {\VNumValue}     {\ruleName{V-NumValue}}
\newcommand {\VZero}         {\ruleName{V-Zero}}
\newcommand {\VTrue}         {\ruleName{V-True}}
\newcommand {\VFalse}        {\ruleName{V-False}}
\newcommand {\VSucc}         {\ruleName{V-Succ}}                 % to be deleted

\newcommand {\BAnno}         {\ruleName{B-Anno}}
\newcommand {\BAnnoFn}       {\ruleName{B-Anno-Fn}}
\newcommand {\BAnnoNonFn}    {\ruleName{B-Anno-Non-Fn}}
\newcommand {\BIFT}          {\ruleName{B-IfTrue}}
\newcommand {\BIFF}          {\ruleName{B-IfFalse}}
\newcommand {\BOp}           {\ruleName{B-Op}}
\newcommand {\BPlus}         {\ruleName{B-Plus}}
\newcommand {\BEq}           {\ruleName{B-Eq}}
\newcommand {\BLet}          {\ruleName{B-Let}}
\newcommand {\BNum}          {\ruleName{B-Num}}
\newcommand {\BVar}          {\ruleName{B-Var}}
\newcommand {\BIF}           {\ruleName{B-If}}
\newcommand {\BTrue}         {\ruleName{B-True}}
\newcommand {\BFalse}        {\ruleName{B-False}}
\newcommand {\BFun}          {\ruleName{B-Fun}}
\newcommand {\BRec}          {\ruleName{B-Rec}}

\newcommand {\BValue}        {\ruleName{B-Value}}
\newcommand {\BIfTrue}       {\ruleName{B-IfTrue}}
\newcommand {\BIfFalse}      {\ruleName{B-IfFalse}}
\newcommand {\BSucc}         {\ruleName{B-Succ}}
\newcommand {\BPredZero}     {\ruleName{B-PredZero}}
\newcommand {\BPredSucc}     {\ruleName{B-PredSucc}}
\newcommand {\BIszeroZero}   {\ruleName{B-IszeroZero}}
\newcommand {\BIszeroSucc}   {\ruleName{B-IszeroSucc}}

\newcommand {\BLetn}         {\ruleName{B-Letn}}
\newcommand {\BLetp}         {\ruleName{B-LetPair}}
\newcommand {\BPair}         {\ruleName{B-Pair}}
\newcommand {\BFst}          {\ruleName{B-Fst}}
\newcommand {\BSnd}          {\ruleName{B-Snd}}
\newcommand {\BFn}           {\ruleName{B-Fn}}
\newcommand {\BApp}          {\ruleName{B-App}}

\newcommand {\Rsectintro}    {\ruleName{$\sectty$intro}}
\newcommand {\Rsectelim} [1] {\ruleName{$\sectty$elim{#1}}}


% ---------------------------------------------------------------------------
% ------------------ Proofs and derivation trees, cases ---------------------
% ---------------------------------------------------------------------------

\newcommand{\proofderiv}[2]{\mathbin{#1  :: #2 }}
\newcommand{\proofderivc}[3]{\proofderiv{#1}{#2 \vdash #3}}
\newenvironment{case}[1]{\paragraph{Case}{#1}\\[1em]}{}
\newenvironment{basecase}[1]{\paragraph{Base case}{#1}\\[1em]}{}
\newenvironment{stepcase}[1]{\paragraph{Step case}{#1}\\[1em]}{}
\newenvironment{subcase}[1]{\textbf{Subcase}{\quad #1}\\[0.5em]}{}

\newcommand{\infera}[3]{\ianc{#3}{#2}{#1}}
\newcommand{\inferaa}[4]{\ibnc{#3}{#4}{#2}{#1}}
\newcommand{\inferaaa}[5]{\icnc{#3}{#4}{#5}{#2}{#1}}

% ---------------------------------------------------------------------------
% --------------------------------- Other -----------------------------------
% ---------------------------------------------------------------------------

\newcommand{\emphFact}[1]{{\color{ForestGreen}{#1}}}

\bibpunct{[}{]}{;}{a}{}{,} % citations style

\newcommand{\bel}{\lstinline}

\newcommand{\rc}[2]{\ensuremath{\mathcal{R}_{#1}(#2)}} % reducibility canadidate


%  LocalWords:  Twelf Bool num FV FMV fn fst snd succ pred iszero eq
%  LocalWords:  Mult Primop


%\usepackage{graphics} % use for the Pitts-Gabbay quantifier
%\usepackage{color}
% \input listing-macros
% \usepackage{todonotes}
% \newcommand{\inlinetodo}[1]{\todo[inline,color=green!40]{#1}}
% \newcommand{\inlinetodoam}[1]{\todo[inline,color=red!40]{#1 -- Alberto}}
% \newcommand{\inlinetodoaa}[1]{\todo[inline,color=green!40]{#1 -- Andreas}}
% \newcommand{\inlinetodobp}[1]{\todo[inline,color=yellow!40]{#1 -- Brigitte}}
%\newcommand{\inlinetodo}[1]{\ednote {#1}} 

%\long\def\ednote#1{\footnote{[{\it #1\/}]}\message{ednote!}}
% \long\def\note#1{\begin{quote}[{\it #1\/}]\end{quote}\message{note!}}

%%%%%%%%%%%%%%%%%%%%%
\newcommand{\sn}[2]{#1 \vdash #2 \in \ensuremath{\mathsf{SN}}}
% \newcommand{\rcs}[2]{\ensuremath{\Delta\vdash\mathcal{R}(#1)\hastype
% #2 }} % reducibility canadidate
\newcommand{\rcs}[3]{\ensuremath{#1 \vdash #2\in \mathcal{R}_{#3} }} % reducibility canadidate



\begin{document}

\conferenceinfo{}{}
\CopyrightYear{}
\copyrightdata{}

\title{POPLMark Reloaded}
\authorinfo
 {Andreas Abel}
 {Department of Computer Science and Engineering, Gothenburg University / Chalmers, Sweden}
 {andreas.abel@gu.se}

 \authorinfo
 {Alberto Momigliano}
 {DI, Universit\`a degli Studi di Milano, Italy }{momigliano@di.unimi.it}
\authorinfo
 {Brigitte Pientka}
 {School of Computer Science, McGill University, Montreal,
           Canada}
 {bpientka@cs.mcgill.ca}
\maketitle

\begin{abstract}
  As a follow-up to the POPLMark Challenge, we propose a new benchmark
  for machine-checked metatheory of programming languages:
  establishing strong normalization of a simply-typed lambda-calculus
  with a proof by Kripke-style logical relations. We believe that this
  case-study overcomes some of the limitations of the original
  challenge and highlights, among others, the need of native support
  for context reasoning and simultaneous substitutions.
\end{abstract}


\category{D.3.1}{Programming Languages}{Formal Definitions and Theory}
\category{F.3.1}
         {Logics and Meanings of Programs}
         {Specifying and Verifying and Reasoning about Programs}
\category{F.4.1}{Mathematical Logic}{Lambda Calculus and Related
  Systems}[Mechanical theorem proving, Proof theory]
% \category{I.2.3}{Artificial Intelligence}{Deduction and Theorem
%   Proving}[Deduction, Inference engines, meta
% theory] \terms{Theory, Languages, Verification}



\keywords
Machine checked meta-theory,
benchmarks,
POPLMark Challenge,
logical frameworks,
strong normalization,
logical relations


\section{Introduction}
\label{sec:intro}
We discuss here an alternative proof method for proving
normalization. We will focus here on a \emph{semantic} proof method
using \emph{saturated sets} (see \cite{Luo:PHD90}). This proof method
goes back to \cite{Girard1972} building on some previous ideas
by \cite{Tait67}.

The key question is how to prove that given a lambda-term, its
evaluation terminates, i.e. normalizes. We concentrate here on a typed
operational semantics following \cite{Goguen:TLCA95} and define
a reduction strategy that transforms $\lambda$-terms into
$\beta$ normal form. This allows us to give a concise presentation of the important issues that arise.

 We see this benchmark as a good jumping point to investigate and mechanize the meta-theory of dependently typed systems where a typed operational semantics simplifies the study of its meta-theory. The approach of typed operational semantics is however not limited to dependently typed systems, but it has been used extensively in studying subtyping,  type-preserving compilation, and shape analysis. Hence, we believe it does describe an important approach to describing reductions.



\section{Simply Typed Lambda Calculus with Type-directed Reduction}
Recall the lambda-calculus together with its reduction rules.


\[
\begin{array}{llcl}
\mbox{Terms}  & M,N & \bnfas & x \mid \lambda x{:A}. M \mid M\;N \\
\mbox{Types} & A, B & \bnfas & \base \mid A \arrow B
\end{array}
\]

We consider as the main rule for reduction (or evaluation) applying a term to an abstraction, called \emph{$\beta$-reduction}.
% together with \emph{$\eta$-expansion}. We only $\eta$-expand a term, if we do not immediately create a redex to avoid infinite alternations between $\eta$-expansion and $\beta$-reduction.
 We use the judgment $\Gamma \vdash M \red N :A$ to mean that both $M$ and $N$ have type $A$ in the context $\Gamma$ and the term $M$ reduces to the term $N$.

% \[
% \begin{array}{lcll}
% %\multicolumn{3}{l}{\mbox{$\beta$-reduction}}  \\
% \Gamma \vdash (\lambda x{:}A.M)\;N : B & \red & \Gamma \vdash [N/x]M : B & \mbox{$\beta$-reduction} \\
% \Gamma \vdash M : A \arrow B & \red & \Gamma \vdash \lambda x{:}A.M~x : A \arrow B & \mbox{$\eta$-expansion}
% \end{array}
% \]

% The $\beta$-reduction rule only applies once we have found a redex.  However, we also need congruence rules to allow evaluation of arbitrary subterms.

\[
\begin{array}{c}
\infer[\beta]
{\Gamma \vdash (\lambda x{:}A.M)~N  \red [N/x]M : B }
%    {\Gamma \vdash \lambda x{:}A.M : A \arrow B & \Gamma \vdash  N : A}
    {\Gamma, x{:}A \vdash M :  B & \Gamma \vdash  N : A}
\qquad
%\infer[\eta]{\Gamma \vdash M \red \lambda x{:}A.M~x : A \arrow B}{
% M \not= \lambda y{:}A.M'}
\\[1em]
\infer{\Gamma \vdash M\,N \red M'\,N : B}{\Gamma \vdash M \red M' : A \arrow B & \Gamma \vdash N : A}
\qquad
\infer{\Gamma \vdash M\,N \red M\,N' : B}{\Gamma \vdash M : A \arrow B & \Gamma \vdash N \red N' : A}
\\[1em]
\infer{\Gamma \vdash \lambda x{:}A.M \red \lambda x{:}A.M' : A \arrow B}{\Gamma, x{:}A \vdash M \red M' : B}
\end{array}
\]

Our typed reduction relation is inspired by the type-directed
definition of algorithmic equality for $\lambda$-terms (see for
example \cite{Crary:ATAPL} or \cite{Harper03tocl}). Keeping track of
types in the definition of equality or reduction becomes quickly
necessary as soon as want to add $\eta$-expansion or add a unit type
where every term of type unit reduces to the unit element. We will
consider the extension with the unit type in Section~\ref{sec:unit}.


On top of the single step reduction relation, we can define multi-step
reductions as usual:

\[
\ianc{\Gamma \vdash M \red N}{\Gamma \vdash M \mred N : B}{} \qquad
\ibnc{\Gamma \vdash M \red N : B}{\Gamma \vdash N \mred M' : B}
      {\Gamma \vdash M \mred M' : B}{}
\]

Our definition of multi-step reductions guarantees that we take at least one step.
In addition, we have that typed reductions are only defined on well-typed terms, i.e. if $M$ steps then $M$ is well-typed.

\begin{lemma}[Basic Properties  of Typed Reductions and Typing]\quad
  \begin{itemize}
  \item If $\Gamma \vdash M \red N : A$ then $\Gamma \vdash M : A$ and $\Gamma \vdash N : A$.
  \item If $\Gamma \vdash M : A$ then $A$ is unique.
  \end{itemize}
\end{lemma}


The typing and typed reduction strategy satisfies weakening and
strengthening\ednote{Should maybe be already formulated using context
  extensions? -bp. Likewise, the subsitution lemma cound be formulated with well typed subst. Moreover the weaken/stren of typing follow from the same for reduction and lemma 1.1 -am}.

\begin{lemma}[Weakening and Strengthening of Typed Reductions]\label{lem:redprop}\quad
  \begin{itemize}
%   \item If $\Gamma, \Gamma' \vdash M : B$ then $\Gamma, x{:}A, \Gamma' \vdash M : B$.
%  \item If $\Gamma, x{:}A, \Gamma' \vdash M : B$ and $x \not\in\FV(M)$ then $\Gamma, \Gamma' \vdash M : B$.
  \item If $\Gamma, \Gamma' \vdash M \red N : B$ then $\Gamma, x{:}A, \Gamma' \vdash M \red N : B$.
  \item If $\Gamma, x{:}A, \Gamma' \vdash M \red N : B$ and $x \not\in \FV(M)$ then
        $x \not\in \FV N$ and $\Gamma, \Gamma' \vdash M \red N : B$.
  \end{itemize}
\end{lemma}
\begin{proof}
By induction on the first derivation.
\end{proof}


\begin{lemma}[Substitution Property of Typed Reductions]\label{lem:redsubst}\quad
If $\Gamma, x{:}A \vdash M \red M' : B$ and $\Gamma \vdash N : A$ then
$\Gamma \vdash [N/x]M \red [N/x]M' : B$.
\end{lemma}
\begin{proof}
By induction on the first derivation, using standard properties of composition of substitutions.
\end{proof}


We also will rely on some standard multi-step reduction properties
which are proven by induction.

\begin{lemma}[Properties of Multi-Step Reductions]\label{lm:mredprop}
\quad
\begin{enumerate}
\item\label{lm:mredtrans} If $\Gamma \vdash M_1 \mred M_2 : B$ and $\Gamma \vdash M_2 \mred M_3 : B$ then $\Gamma \vdash M_1 \mred M_3 : B$.
\item\label{lm:mredappl} If $\Gamma \vdash M \mred M' : A \arrow B$
  and $\Gamma \vdash N : A$ 
  then $\Gamma \vdash M~N \mred M'~N : B$.
\item\label{lm:mredappr} If $\Gamma \vdash M : A \arrow B$ and $\Gamma \vdash N \mred N' : A$ then $\Gamma \vdash M~N \mred M~N' : B$.
\item\label{lm:mredabs} If $\Gamma,x{:}A \vdash M \mred M' : B$ then $\Gamma \vdash \lambda x{:}A.M \mred \lambda x{:}A.M' : A \arrow B$.
\item\label{lm:mredsubs} If $\Gamma, x{:}A \vdash M : B$ and $\Gamma \vdash N \red N' : A$
then $ \Gamma \vdash [N/x]M \mred [N'/x]M : B$.
\end{enumerate}
\end{lemma}



\subsection*{When is a term in normal form?}

We define here briefly when a term is in $\beta$-normal form.
% The presence of $\eta$ again requires our definition to be type directed.
We define the grammar of normal terms as given below

\[
\begin{array}{llcl}
\mbox{Normal Terms}  & M,N & \bnfas & \lambda x{:A}. M \mid R \\
\mbox{Neutral Terms} & R, P & \bnfas & x \mid R\;M \\
  \end{array}
\]

This grammar does not enforce $\eta$-long.
% For example, $\lambda x{:}A \arrow A. x$ is not in $\eta$-long form.
% To ensure we only characterize $\eta$-long forms, we must ensure that we allow to switch between normal and neutral types at base type.  % On the other hand, $\lambda x{:}A \arrow A. \lambda y{:}A.x~y$ is in $\beta$-short and $\eta$-long form.

% \[
%   \begin{array}{c}
% \multicolumn{1}{l}{\fbox{$\nf {\Gamma \vdash M} A$}~~\mbox{Term $M$ is normal at type $A$}}\\[1em]
% \ianc{\nf {\Gamma, x{:}A \vdash M} B}
%      {\nf {\Gamma \vdash \lambda x{:}A.M} {A \arrow B}}{} \quad
% \ianc{\neu {\Gamma \vdash R}{\base}}
%      {\nf {\Gamma \vdash R}{\base}}{}
% \\[1em]
% \multicolumn{1}{l}{\fbox{$\neu {\Gamma \vdash M} A$}~~\mbox{Term $M$ is neutral at type $A$}}\\[1em]
% \ibnc{\neu {\Gamma \vdash R} {A \arrow B}}{\nf {\Gamma \vdash M} A}
%      {\neu {\Gamma \vdash R~M} {B}}{}
% \qquad
% \ianc{x{:}A \in \Gamma}{\neu {\Gamma \vdash x} {A}}{}
%   \end{array}
% \]

% In practice, it often suffices to enforce that we reduce a term to a weak head normal form. For weak head normal forms we simply remove the requirement that all terms applied to a neutral term must be normal.




\subsection*{Proving normalization}
The question then is, how do we know that we can normalizing a well-typed lambda-term into its $\beta$ normal form? - This is equivalent to asking whether after some reduction steps we will end up in a normal form where there are no further reductions possible. Since a normal lambda-term characterizes normal proofs, normalizing a lambda-term corresponds to normalizing proofs and demonstrates that every proof in the natural deduction system indeed has a normal proof. %

Proving that reduction must terminate is not a simple syntactic argument based on terms, since the $\beta$-reduction rule may yield a term which is bigger than the term we started with. % Further, $\eta$-expansion might make the term bigger.

As syntactic arguments are not sufficient to argue that we can always compute a $\beta$ normal form, we hence need to find a different inductive argument. For the simply-typed lambda-calculus, we could prove that while the expression itself does not get smaller,  the type of an expression does\footnote{This is the essential idea of hereditary substitutions \cite{Watkins02tr}}.  This is a syntactic argument; it however does not scale to polymorphic lambda-calculus or full dependent type theories. We will here instead discuss a \emph{semantic} proof method where we define the meaning of well-typed terms using the abstract notion of \emph{reducibility candidates}.

Throughout this tutorial, we stick to the simply typed lambda-calculus and its extension. This allows us to give a concise presentation of the important issues that arise.  However the most important benefits of typed operational semantics and our approach are demonstrated in systems with dependent types  where our development of the metatheoretic simpler than the existing techniques. We see this benchmark hence as a good jumping point to investigate and mechanize the meta-theory of dependently typed systems.


% Unlike all the previous proofs which were syntactic and direct based on the structure of the derivation or terms, semantic proofs

\section{Semantic Interpretation}
Working with well-typed terms means we need to be more careful to
consider a term within its typing context. In particular, when we
define the semantic interpretation of $\inden{\Gamma}{M}{A \arrow B}$
we must consider all extensions of $\Gamma$ (described by $\Gamma'
\ext \rho \Gamma$) in which we may use $M$.

\begin{itemize}
\item $\inden{\Gamma}{M}{\base}$ iff $\Gamma \vdash M\hastype \base$ and $M$ is strongly normalizing
% , i.e. $\Gamma \vdash M \in \SN$.
\item $\inden{\Gamma}{M}{A \arrow B}$ iff for all $\Gamma' \ext{\rho} \Gamma$ and $\Gamma' \vdash N :A$, if $\inden{\Gamma'}{N}{A}$ then $\inden{\Gamma'}{[\rho]M~N}{B}$.
\end{itemize}


% Weakening holds for the semantic interpretations.

% \begin{lemma}[Semantic Weakening]\ref{lm:sweak}
% If $\Gamma \models M : A$ then $\Gamma, x{:}C \models M : A$.
% \end{lemma}

% We sometimes write these definitions more compactly as follows

% \[
% \begin{array}{llcl}
% \mbox{Semantic base type} & \den{o} & := & \SN  \\
% \mbox{Semantic function type} & \den{A \arrow B} & := & \{ M | \forall \Gamma' \ext{\rho} \Gamma,~\forall \Gamma' \vdash N : A.~ \Gamma'\ models N : A \ \in \den{A}. M\;N \in \den{B} \}
% \end{array}
% \]


\section{General idea}

We prove that if a term is well-typed, then it is strongly normalizing in  two steps:

\begin{description}
\item[Step 1] If $\inden{\Gamma}{M}{A}$ then $\Gamma \vdash M : A$ and $M$ is strongly normalizing.
\item[Step 2] If $\Gamma \vdash M : A$ and $\inden{\Gamma'}{\sigma}{\Gamma}$ then $\inden{\Gamma'}{[\sigma]M}{A}$.
\end{description}

Therefore, we can conclude that if a term $M$ has type $A$ then $M$ is strongly normalizing and its reduction is finite, choosing $\sigma$ to be the identity substitution.
% \\[1em]
% We remark first, that all variables are in the semantic type $A$ and variables are strongly normalizing, i.e. they are already in normal form.

% % \begin{lemma}~\\
%   \begin{itemize}
%   \item If $\Gamma \vdash x : A$ then $\Gamma \models x : A$
%   \item If $\Gamma \vdash x : A$ then $(\Gamma \vdash x) \in \SN$.
%   \end{itemize}

% \end{lemma}

% These are of course statements we need to prove.
\section{The Challenge}
\label{sec:chal}
% \begin{metanote}
%   Intentionally using same sections of POPLMark
% \end{metanote}

\subsection{Problem Selection}
\label{ssec:select}

(Strong) normalization by Tait's method is a well-understood and
reasonably circumscribed problem that has been a cornerstone of
mechanized PL theory, starting from~\citep{Altenkirch93}. There are of
course many alternative ways to prove SN for a lambda-calculus, see
for example the inductive approach of~\citep{Joachimski2003},
partially formalized in~\citep{ABEL20083}, or by reduction from strong
to weak normalization~\cite{SORENSEN199735}. For that matter, a SN
proof via logical relations for the simply-typed lambda calculus can
be carried out (see for a classic example~\citep{girardLafontTaylor})
{without} appealing to a Kripke definition of reducibility, at the
cost, though, of a rather cavalier approach to ``free''
variables. However, the Kripke technique is handy in establishing SN for richer
theories such as dependently typed ones, as well as for proving
stronger results, for example about equivalence
checking~\citep{Crary:ATAPL,Harper03tocl}.




We claim that mechanizing such a proof is
indeed challenging since:

\begin{itemize}
\item It focus on reasoning on \emph{open} terms and on relating
  different contexts or \emph{worlds}, taking seriously the Kripke
  analogy. The quantification over \emph{all} extensions of the given
  world may be problematic for frameworks where contexts are only
  implicitly represented, or, on the flip side, may require several
  boring weakening lemmas in first-order representations.
\item The definition of \emph{reducibility} requires a sophisticated
  notion of inductive definition, which must be compatible with the
  binding structures, but also be able to take into account
  \emph{stratification}, to tame the negative occurrence of the
  defined notion.
\item Simultaneous substitutions and their equational theory
  (composition, commutation etc.) are central in formulating and
  proving the main result. For example, in the proof of the Fundamental
  Theorem~\ref{thm:fund}, we need to push substitutions through
  (binding) constructs.
\end{itemize}
%
In this sense, this challenge goes well beyond the original PC, where
the emphasis was on binder representations, proofs by structural
induction and operational semantics animation.


\smallskip

Previous formalizations of strong normalization usually follows
Girard's approach, see for example~\cite{DonnellyX07} carried out in
ATS/LF, or the one available in the Abella repository
(\url{abella-prover.org/~normalization/}).  Less frequent are
formalizations following the Kripke discipline: both~\cite{CaveP15}
and~\cite{NarbouxU08} encode~\cite{Crary:ATAPL}'s account of decision
procedures for term equivalence in the STLC, in Beluga and Nominal
Isabelle respectively; the latter was then extended
in~\citep{Urban2011} to formalize the analogous result for
LF~\citep{Harper03tocl}. See~\citep{AbelV14} for a SN Kripke-style
proof for a more complex calculus and~\citep{Rabe:2013} for another
take to handling dependent types --- this paper also contains many
more references to the literature.


%% this bit is new

The choice of a Kripke-style proof of SN for the STLC may sound
contentious on several grounds and hence we will try to motivate it
further:

\begin{itemize}
\item We acknowledge that SN is not the most exciting application of logical
  relations, some of which we have mentioned in the previous Section. Still, it
  is an important topic in type theory, in particular w.r.t.\ logical
  frameworks' meta-theory, see for example~\citep{AltenkirchK16}, and
  in this sense dear to our hearts. It is  a well-known textbook example,
  which uses techniques that should be familiar to the community of
  interest in the simplest possible setting.
\item Yes, the STLC is the prototypical toy language, while a
  POPL paper will address richer PL theory
  aspects. For one, adding more constructs, say in the PCF direction,
  perhaps with an iterator, would make the proof of the fundamental
  theorem longer, but not more interesting. Secondly, we think that a
  good benchmark should be simple  enough that it
  could be tried out almost immediately if one is acquainted with
  proof-assistants. Conversely, it should encourage a PL theorist to
  start playing with proof assistants. Finally, we do suggest extensions
  of our challenge in the next Section.
\item The requirement of the ``Kripke-style'' may seem overly
  constrictive, especially since this may  not be strictly needed for
  the STLC\@. However, as we have argued before, this is meant as a
  springboard for more complex case studies, where this technique
  is forced on us. Remember that we are interested in \emph{comparing} solutions. A more
  ambitious challenge may not solicit enough solutions, if the problem
  is too exotic or simply too lengthy.
\end{itemize}



% Therefore we insist that the proof must be Kripke-style, others do not
% apply!
\subsection{Evaluation Criteria}
\label{ssec:ev}
One of the limitations of the PC experiment was in the
\emph{evaluation} of the solutions, although it is not easy to avoid
the ``trip to the zoo'' effect, well-known from trying to comparing programming
languages: there is no theory underlying the evaluation; criteria
tend to be rather qualitative, and finally, the comparison itself may
be lengthy~\citep{companion}. Within these limitations, of the
proposed solutions we will take into consideration the:
\begin{itemize}
\item  Size of the necessary infrastructure for defining the base language:
    binding, substitutions, renamings, contexts, together with
    substitution and other infrastructural lemmas.
  \item Size of the main development versus the main theorems in the
    on-paper proof, in particular, number of technical lemmas not
    having a direct counterpart in the on-paper proof.
\end{itemize}
More qualitatively, we will try to assess the:
\begin{itemize}
\item Ease of using the infrastructure for supporting binding,
  contexts, etc. How easy is it to apply the appropriate lemmas in the
  main proof? For example, does applying the equational theory of
  substitutions require low-level rewriting, or is it automatic?
\item Ease of development of the overall proof; what support is
  present for proof construction, when not for proof and counterexample
  search?
\end{itemize}
\subsection{The Challenge, Explained}
\label{ssec:expl}






Let us recall the definition of the STLC, starting with the grammar of terms, types, contexts and substitutions:
\[
\begin{array}{ll@{\bnfas}l}
\mbox{Terms} & M, N & x \bnfalt \lam x{:}T.M \bnfalt M \app N \\
\mbox{Types} & T, S & B \bnfalt T \arrow S\\
%%\mbox{Values} & V & \lam x.M 
\mbox{Context} & \Gamma & \cdot\bnfalt\Gamma, x\oftp T\\
\mbox{Subs} & \sigma & \epsilon\bnfalt\sigma, N/x
\end{array}
\]
The static and dynamic semantics are standard and are depicted in
Figure~\ref{fig:stlc}. Since we want to be very upfront about the fact
that evaluation goes under a lambda and thus involves open terms, we
make the context explicit even in the reduction rules, contrary to
what, say, Barendregt would do. Note that, because of rule \EAbsStep,
we do not need to assume that the base type is inhabited by a
constant.  We denote with $[\sigma] M$ the application of the
simultaneous substitution $\sigma$ to $M$ and with
$[\sigma_1]\sigma_2$ their composition.
% \unsure{going for church typing
% to have one form of ctx even in step -am}
\begin{figure*}[t!]
  \centering
  \[
\begin{array}{c}
\multicolumn{1}{l}{\fbox{$\Gamma \vdash \tmhastype M
    T$}\quad\mbox{Term $M$ has type $T$ in context $\Gamma$} }
\\[1em]
\infer[u]{\Gamma \vdash \tmhastype x T}{\tmhastype x T \in \Gamma} \qquad
\infer[\TFn^{x}]{\Gamma \vdash \tmhastype {(\lam x\oftp T.M)} {(T \arrow S)}}
                 {\Gamma, \tmhastype x T \vdash \tmhastype M S}
\qquad %\\[1em]
\infer[\TApp]{\Gamma \vdash \tmhastype {(M \app N)} S}
             {\Gamma \vdash \tmhastype M (T \arrow S)
  & \Gamma \vdash \tmhastype N T}\\[1em]
\end{array}
\]
%
\[
\begin{array}{c}
\multicolumn{1}{l}{\fbox{$\Gamma \vdash M \Steps M'$}\quad\mbox{Term $M$ steps to term $M'$ in  context $\Gamma$}}
\\[1em]
\infer[\EAbsStep^x]{\Gamma \vdash\lam x\oftp T.M \Steps \lam x\oftp T.M'}{\Gamma,x\oftp T \vdash M \Steps M'} \quad
\infer[\EAppBeta]{\Gamma \vdash (\lam x\oftp T.M) \app N \Steps [N/x]M}{} \quad %\\[1em]
\infer[\EAppArgStep]{\Gamma \vdash M \app N \Steps M'\;N}{\Gamma \vdash M \Steps M'} \quad
\infer[\EAppFnStep]{\Gamma \vdash M \app N \Steps M\;N'}{\Gamma \vdash N \Steps N'}
% \\[1em]
% \multicolumn{1}{l}{\fbox{$M \MSteps M'$}~~ \mbox{Term $M$ steps in
%     multiple steps to term $M'$}}\\[1em]
% \infer[\MRef]{M \MSteps M}{} \qquad
% \infer[\MOne]{M \MSteps M'}{M \Steps N & N \MSteps M'} 
\end{array}
\]

  \caption{Typing and reduction rules for the STLC}
  \label{fig:stlc}
\end{figure*}

We now define the set of \emph{strongly-normalizing} terms as
pioneered by~\cite{Altenkirch93} and by now usual:
\[
\infer[\mathit{SN-WF}]{\sn \Gamma M}{\forall M'.~\Gamma\vdash M\Steps M'\quad \sn \Gamma {M'}}
\]
expressing that the set of strongly normalizing terms is the
well-founded part of the reduction relation. A more explicit
formulation of strong normalization is allowed, see for
example~\citep{Joachimski2003}, but then an equivalence proof should
be provided. Note that reasoning with the above rule \textit{SN-WF}
cannot proceed by structural induction, since it is not the case that
$M'$ is a sub-term of $M$.
%

The logical predicates have the following structure:
\begin{itemize}
\item  $\rcs \Gamma M T$, and
\item $\rcs {\Gamma'} \sigma \Gamma$.
\end{itemize}


We use a Kripke-style logical relations definition where we
\emph{witness} the context extension using a \emph{weakening} substitution
$\rho$. This can be seen as a \emph{shift} in de Bruijn terminology, while 
%% something about shifting in DB, no need in named approaches 
% This treatment is inspired by Beluga (see,
% e.g.,~\citep{CaveP15}) to underline the context reasoning, for which
% $M$ depends on variable in $\Gamma$ and needs to be weakened to live
% in world $\Delta$.
other encodings may use different (or no particular) implementation
 techniques for handling context extensions.


\begin{definition}
 [Reducibility Candidates]
\mbox{}
\begin{itemize}
\item $\rcs \Gamma M B$ iff $\Gamma\vd M\hastype B$ and $\sn \Gamma M$:
\item $\rcs \Gamma M {T\arrow S}$ iff
  $ \Gamma\vd M\hastype {T\arrow S}$ and for all $N,\Delta$ such that
  $\Gamma \leq_\rho \Delta$, if $\rcs \Delta N {T}$ then
  $\rcs \Delta {([\rho]M) \app N} {S}$.
\end{itemize}
\end{definition}


As usual, we lift reducibility to substitutions:
\begin{definition}[Reducible Substitutions]\mbox{}
  \begin{itemize}
  \item \rcs {\Gamma'} {\epsilon} \cdot
\item $\rcs {\Gamma'} {\sigma, N/x}  {\Gamma, x:T} $ % \\ 
  iff 
$\rcs  {\Gamma'} \sigma \Gamma $ and  
$\rcs  {\Gamma'} N T$.
  \end{itemize}
\end{definition}

%We use $\rho$ for a renaming substitution. 
We now give an outline of the proof as a sequence of lemmas --- the
reader will find all the details in the forthcoming full version of this paper.

\begin{lemma}[Semantic Function Application]\mbox{}\\
If $\rcs {\Gamma} {M} {T \arrow S}$
and $\rcs \Gamma N T$
%\\
then $\rcs \Gamma {M~N} S$.
\end{lemma}
\begin{proof}
  Immediate, by definition.
\end{proof}

\begin{lemma}[SN Closure under Weakening]
\label{le:sn-clo}\mbox{} \\
If $\Gamma_1\leq_\rho \Gamma_2$ % and $\Gamma_1 \vdash M : A$
 and $\sn {\Gamma_1} M$ 
then $\sn {\Gamma_2} {[\rho]M}$. 
\end{lemma}
\begin{proof}
By induction on the derivation of $\sn {\Gamma_1} M$.  
\end{proof}

\begin{lemma}[Closure of Reducibility under Weakening]\mbox{} \\
If $\Gamma_1 \leq_\rho \Gamma_2$ and 
 $\rcs {\Gamma_1} M T$ then $\rcs {\Gamma_2} {[\rho]M} T$. 
\end{lemma}
\begin{proof}
  By cases on the definition of reducibility using the above
  Lemma~\ref{le:sn-clo} and weakening for typing.
\end{proof}

\begin{lemma}[Weakening of Reducible Substitutions]\label{lem:weakredsub}\mbox{} \\
If $\Gamma_1 \leq_\rho \Gamma_2$ and 
  $\rcs {\Gamma_1} \sigma \Phi$ 
then $\rcs {\Gamma_2} {[\rho]\sigma} \Phi$.
\end{lemma}
\begin{proof}
  By induction on the derivation of $\rcs {\Gamma_1} \sigma \Phi$
  using Closure of Reducibility under Weakening.
\end{proof}
% \[
%   \begin{array}{lcl}
% \mbox{Evaluation Context}~E  & \bnfas & [~~] \mid E~M
%   \end{array}
% \]

% \begin{lemma}[Closue under Weak Head Expansion]\mbox{}\\
% If $\sn \Gamma N$ and 
% $\rcs \Gamma {E[~[N/x]M~]} T$ \\
% then 
% $\rcs \Gamma {E[~(\lambda x{:}A.M)~N~]} T$  
% \end{lemma}


\begin{lemma}[Closure under Beta Expansion]\label{lem:betaclosed}\mbox{} \\
If $\sn \Gamma N$ 
and $\rcs \Gamma {[N/x]M} S$ %\\
then $\rcs \Gamma {(\lambda x{:}T.M)~N} S$.
\end{lemma}
\begin{proof}
  By induction on $S$ after a suitable generalization.
\end{proof}
%
\begin{theorem}[Fundamental Theorem]
\label{thm:fund}\mbox{} \\
If $\Gamma  \vdash M : T$ 
and $\rcs {\Gamma'} \sigma \Gamma$ 
% \\ 
then $\rcs {\Gamma'} {[\sigma]M} T$.
\end{theorem}
\begin{proof}
  By induction on $\Gamma \vdash M : T$. In the case for functions, we
  use Closure under Beta Expansion (Lemma~\ref{lem:betaclosed}) and
  Weakening of reducible substitution (Lemma~\ref{lem:weakredsub}).
\end{proof}



%%% Local Variables:
%%% mode: latex
%%% TeX-master: "poplr"
%%% End:

%  LocalWords:  TODO Kripke Girard's Matthes Gandy Girard Abella LF
%  LocalWords:  DonnellyX LICS tocl CaveP TAPL Moggi’s renamings STLC
%  LocalWords:  Barendregt iff Schurmann Tait's Altenkirch Joachimski
%  LocalWords:  SORENSEN girardLafontTaylor Crary ATAPL AbelV Rabe de
%  LocalWords:  POPL ICFP Bruijn reducibility AltenkirchK PCF WF


\section{Beyond the Challenge}
\label{sec:beyond}
There is an ongoing tension between weak and strong logical
frameworks~\citep{DeBruijn91lf}, with which we can encode our
benchmarks. Weak frameworks are designed to accommodate advanced
infra-structural features for binders (HOAS/nominal syntax etc.) and
for judgments (hypothetical and parametric), but may struggle on other
issues, such as facilities for computation or higher-order
quantification/impredicativity. There are at least two coordinates in
which we can directly extend our benchmark, to further highlight this
dilemma:
\begin{itemize}
\item Logical relations for dependent types~\citep{Rabe:2013,AbelV14},
  up to the Calculus of Constructions. Here we need to go beyond
  first-order quantification, which is typically what is on offer in
  weak frameworks.
\item Proof by logical relation via
  \emph{step-indexing}~\citep{Appel:2001}. Here we have two issues:
  \begin{enumerate}
  \item the logical relation may even be harder to be accepted by the
    meta-language as an inductive definition 
% in its original   formulation 
than with simple types; in fact, the work around the
    negative occurrence of the defined relation cannot be based on
    structural induction on types, but it has to use some form of
    course-of-value induction.

\item   It involves a limited amount of arithmetic reasoning:
  \begin{quote}
    ``definitions and proofs have a tendency to become
  cluttered with extra indices and even arithmetic, which are really
  playing the role of construction line.'' (\cite{BentonH10}).
  \end{quote}
  This latter point may be problematic for frameworks such as Abella and Beluga,
  which do not (yet) have extensive libraries, nor computational mechanisms
  (rewriting, reflection) for those  tasks.
  \end{enumerate}
\end{itemize}

\section{Call for action}

We ask the community to submit solutions and 
%  - Give a short (informal) talk at LFMTP to present their solution
we plan to invite everyone who does so to contribute
towards a joint paper discussing trade-offs between them. The authors
commit themselves to produce solutions in Agda,
Abella and Beluga. To resurrect the slogan from the PC, a small step
(excuse the pun) for us, a big step for bringing mechanized
meta-theory to the masses!


%%% Local Variables:
%%% mode: latex
%%% TeX-master: "poplr"
%%% End:

%  LocalWords:  DeBruijn lf Appel Coq Agda Abella HOAS Rabe
%  LocalWords:  impredicativity


\bibliographystyle{abbrvnat}
\bibliography{p}
% \appendix

% \section{Appendix: Sketch of the Proof}
% \label{sec:app}


\end{document}

%%% Local Variables:
%%% mode: latex
%%% TeX-master: t
%%% End:

%  LocalWords:  POPLMark Andreas Gothenburg Chalmers Momigliano degli
%  LocalWords:  Universit Studi di Milano Pientka McGill metatheory
%  LocalWords:  Kripke
